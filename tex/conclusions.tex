\documentclass{article}
\begin{document}
\section*{Conclusions}
It is possible to design linear filters for seasonal adjustment of time series that are more effective for real-time applications in official statistics than the filters currently in widespread use. Most packages already provide several options for filter coefficients, which means that it is straightforward to add the filter presented in this paper as anotheralternative.\\The advantage of the filter presented here is that the coefficients are non-zero over a range that is rather more restricted than the one obtained when recursively appying the existing naive options. This means that the resulting time series with seasonal adjustments can be finalised faster, and extrapolation towards the end points of time series is better constrained since it needs to take place over a much more limited range.\\An additional benefit of filtering all signals in a frequency band, is that this adjusts to a large degree for slow modulation of the amplitude of seasonal effects, as well as compensating irregularities in monthly sampling which may occur because in practice the sampling is done eg. on the last friday or the last working day of the month even though it is treated as if it were perfectly regular.\\The linearity of the filter together with the non-recursive formulation of it implies that partial time series can be adjusted for seasonal influences separately and then totalised which will by construction yield identical results to the seasonal adjustment of the summation of the partial time series.\\A JDemetra+ analysis of the output of this filtering operation confirms the aimed achievements.
\end{document}